\documentclass[a4paper]{article}
%\usepackage{simplemargins}

%\usepackage[square]{natbib}
\usepackage{amsmath}
\usepackage{amsfonts}
\usepackage{amssymb}
\usepackage{graphicx}

\begin{document}
\pagenumbering{gobble}

\Large
 \begin{center}
Predator-prey network temporal dynamics in extreme environments\\ 

\hspace{10pt}

% Author names and affiliations
\large
Alejandra J. Troncoso \\

\hspace{10pt}

\small  
Departamento de Biología, Insituto de Ecología y Biodiversidad\\
matroncoso@userena.cl\\
\end{center}

\hspace{10pt}

\normalsize

Long-Term Ecological Research aims to monitor long-term ecological changes to develop knowledge and predictive understanding to conserve, protect, and manage ecosystems’ biodiversity. The Fray Jorge LTER is the longest of its kind in South America and is located within the Bosque Fray Jorge National Park (71°40’W, 30°38’S, ca. 9000 ha) which is a recognized World Biosphere Reserve about 380 km N of Santiago, Chile, and 150 km S of the southern edge of the Atacama Desert. Throughout the 30+ years of monitoring, we have collected biotic and abiotic data before, after, and in-between repeated climatic disturbance episodes (ENSO and high rainfall events) of variable intensity, an optimal opportunity for scientific insight.

Fluctuations in the abundance of resources due to extreme environmental change trigger dramatic changes in community and food web size structure (Woodward et al. 2012), and one of the critical challenges to testing hypotheses about the potential effects of extreme climatic variability on ecological interactions, such as food webs, is the acquisition of empirical data over multiple cycles and at spatial and temporal scales that capture essential climatic variation. Such data is available from the Fray Jorge LTER since we collected and analyzed the composition of regurgitated pellets from three top predators (Tyto alba, Athene cunicularia, and Bubo virginianus) over 25 years (1989-2015); and by means of an ecological network approach, we reconstructed monthly bipartite networks between these three predators and their prey for the 25-year span. we aim to quantify the effects of climatic variability and resource availability on food web topology and stability through multiscale temporal variation assessment and variance partitioning of the dynamics of predator-prey network topology.


\end{document}