\documentclass[a4paper]{article}
%\usepackage{simplemargins}

%\usepackage[square]{natbib}
\usepackage{amsmath}
\usepackage{amsfonts}
\usepackage{amssymb}
\usepackage{graphicx}

\begin{document}
\pagenumbering{gobble}

\Large
\begin{center}
3D HIGH-RESOLUTION TRAIT: IMPROVING FUNCTIONAL DIVERSITY ESTIMATION IN TROPICAL ECOSYSTEMS\\ 

\hspace{10pt}

% Author names and affiliations
\large
Nidia Noemi Fabré, Vandick Batista \\

\hspace{10pt}

\small  
Federal University of Alagoas, Brazil.\\


\end{center}

\hspace{10pt}

\normalsize

The ecomorphological approach uses individuals' morphometric data to proxy ecological variables. To this end, ecomorphology depends on proving relationships between morphology, function, and ecology, as well as estimating and evaluating how the assumptions of the form-function-phylogeny-ecology relationship can impact functional ecology studies. We have evidence that the resolution of functional traits affects the functional diversity estimation of tropical marine fish communities, and the fish body ecomorphology is not accurate for describing tropical functional diversity (FD). On the other hand, the morphometry of internal anatomical structures may represent more assertive functional traits for estimating FD in the tropics, which has been qualified as highly redundant. Otoliths are calcified internal anatomical structures of the inner ear of bony fish. They are formed in the embryonic stages and accumulate calcium throughout the life cycle, changing shape and registering growth rings daily and seasonally. The otoliths help equilibrium and detect changes in orientation and gravitational fish. For a long time, otoliths have been used to indicate fish characteristics, including their ecological diversity. The anatomy of two-dimensional semicircular canals in the inner ear and their otoliths has been used to indicate the diverse functions of species and guilds. Since these structures are phylogenetically and chemically conservative, many studies have been developed to trace a relationship with the ecological functions of species. In the last decades, we have used indices, Fourier harmonics or Wavelets (which allow a “finer” analysis of variations in the shape of the otoliths). However, today we need to discover new indicators that offer a better resolution of the ecological functions of these structures. Aspects of the otoliths, such as sulcus depth and the surface of contact with nerve cells, can be better-explained three-dimensionally. Thus, new technologies, such as the use of computed tomography to describe the brain of the fish or the otolith, and the use of scanning electron microscopy to detail the surface of the otoliths, are techniques that can bring new information on the functional role of species in ecosystems. Our challenge is to transform the diversity of high-resolution three-dimensional shapes into coefficients that can be introduced into the matrices of functional traits. High-resolution traces can generate accurate estimates of the functional structure of megadiverse communities in the tropics, where the compaction of ecological niches requires high-resolution traces, overturning the often-cited high functional redundancy hypothesis for the tropics.

\end{document}
