\documentclass[a4paper]{article}
%\usepackage{simplemargins}

%\usepackage[square]{natbib}
\usepackage{amsmath}
\usepackage{amsfonts}
\usepackage{amssymb}
\usepackage{graphicx}

\begin{document}
\pagenumbering{gobble}

\Large
 \begin{center}
    Interdisciplinary themes on building conservation in extreme environmental scenarios\\ 

\hspace{10pt}

% Author names and affiliations
\large
Vandick Batista

\hspace{10pt}

\small  
University of Alagoas, Brazil\\

\end{center}

\hspace{10pt}

\normalsize

Usual ecological knowledge focuses on information related to the attributes affecting non-human systems, mainly on species distribution and abundance. However,  socio environmental systems require a broad approach, involving transcultural, economic and psychosocial information to allow the management of renewable resources on a participative approach. New modelling approaches that consider human perceptions, attitudes and behaviour dealing with natural resources are desirable to allow effective management of conservation and the sustainable use of those resources. Here we will show alternatives to incorporate interdisciplinary attributes on systems models allowing potential improvement on management approaches.

"Population versus community approach to dealing with tropical fisheries" 

Resource management is typically supported by monospecific attributes allowing precise predictions facing different exploitation levels. However, tropical environments have high diverse biota that makes tradicional approach too expensive and time consuming to support management decisions. More than goals, precise predictions or estimation of population levels, refocus on trends and security margins are traditional successful measures to local communities intuitive management, which is a reference to research refocusing on tendencies on resource abundance. Here we intend to show local ecological knowledge as a reference to management supportive research.\end{document}