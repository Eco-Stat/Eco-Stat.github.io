\documentclass[a4paper]{article}
%\usepackage{simplemargins}

%\usepackage[square]{natbib}
\usepackage{amsmath}
\usepackage{amsfonts}
\usepackage{amssymb}
\usepackage{graphicx}

\begin{document}
\pagenumbering{gobble}

\Large
 \begin{center}
Diversity is not heterogeneity, and the difference matters \\ 

\hspace{10pt}

% Author names and affiliations
\large
Aaron Ellison $^1$, Ayelet Shavit$^2$ \\

\hspace{10pt}

\small  
$^1$ Harvard University, USA\\
$^2$ Tel Hai College, Israel

\end{center}

\hspace{10pt}

\normalsize

Difference is fundamental for building our most basic categories, including socioecological and political systems, models, and their causal explanations; ecologists and statisticians alike devote much effort to observing, describing, quantifying, and analyzing differences. We explicate and formalize two different meanings of "difference": "diversity" and "heterogeneity". The statement "a zoo is diverse whereas an ecosystem is heterogenous" encapsulates our point. The same number of species and individual animals could exist in the same spatial proximity in either a zoo or an ecosystem, but interactions among species or individuals affects only an ecosystem. We argue that "diversity" does not describe well collectives—including ecological communities or ecosystems—whereas "heterogeneity" more appropriately describes collectives and should be limited to describing them. We further argue that ignoring this distinction in the meaning of "difference" and sufficing with measures of "diversity" for models of collective phenomena (e.g., ecologic, epistemic, social, or evolutionary groups) is a value-laden choice with non-trivial epistemic, moral, and environmental results.

\end{document}
