\documentclass[a4paper]{article}
%\usepackage{simplemargins}

%\usepackage[square]{natbib}
\usepackage{amsmath}
\usepackage{amsfonts}
\usepackage{amssymb}
\usepackage{graphicx}

\begin{document}
\pagenumbering{gobble}

\Large
\begin{center}
Population versus community approach to dealing with tropical fisheries\\ 

\hspace{10pt}

% Author names and affiliations
\large
Vandick Batista

\hspace{10pt}

\small  
University of Alagoas, Brazil\\

\end{center}

\hspace{10pt}

\normalsize

Resource management is typically supported by monospecific attributes allowing precise predictions facing different exploitation levels. However, tropical environments have high diverse biota that makes tradicional approach too expensive and time consuming to support management decisions. More than goals, precise predictions or estimation of population levels, refocus on trends and security margins are traditional successful measures to local communities intuitive management, which is a reference to research refocusing on tendencies on resource abundance. Here we intend to show local ecological knowledge as a reference to management supportive research.
\end{document}