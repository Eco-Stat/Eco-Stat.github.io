\documentclass[a4paper]{article}
%\usepackage{simplemargins}

%\usepackage[square]{natbib}
\usepackage{amsmath}
\usepackage{amsfonts}
\usepackage{amssymb}
\usepackage{graphicx}

\begin{document}
\pagenumbering{gobble}

\Large
 \begin{center}
ALTERNATIVE METHODS TO EVALUATE THE GROWTH AND SHAPE OF OTOLITHS \\ 
\hspace{10pt}

% Author names and affiliations
\large

Paulo Duarte-Neto\\

\hspace{10pt}

\small  
Departamento de Estatística e Informática, Universidade Federal Rural de Pernambuco, Recife-PE, Brazil.\\
email-paulo.duartent@ufrpe.br\\


\end{center}

\hspace{10pt}

\normalsize

Otoliths represent the "black box" of teleost fishes, i.e., they function as an encrypted source of life history, demographic, and ecologic information, and are considered indispensable in the fish stock evaluation and management practice. They are calcified concretions found in the fish's inner ear and are associated with hearing, balance, and orientation functions. It is well known that the shape of sagittal otolith varies widely from simple circular or ellipsoidal forms to rather complex patterns, where a pronounced species dependency is observed. Questions about otolith shape, size, growth, and diversity are common for biologists and neuroscientists, held from different perspectives. For example, the former would be interested in the shape of otoliths to explain possible differences between species and populations. The latter would be interested in the functional significance of a specific otolith shape. One of the questions still open and common to these two areas would be the reason for different shapes in otoliths. To date, diverse techniques have been employed to quantify this variability, were landmarks, elliptical Fourier analysis, and Wavelet analyses appear to be the most used techniques.

The objects of classical shape analysis are composed of compact differentiable manifolds, smooth curves, or surfaces that include their boundaries. In this view, natural contours consist of a superficial coating of texture or irregularity attached to a compact underlying structure. For instance, rough contours can be decomposed into smooth differentiable trends and rough additions. In this context, it will be shown in this talk that the morphology of the otoliths goes beyond their ordinary smoothed shape, and this broader look can reveal meaningful and sometimes unexpected information. The discussion will be based on the main results of the project AIMMO (Image Analysis and Modelling of Otolith Growth), in which alternative techniques such as Multifractal Analysis, Topological Data Analysis, Complex Functional Representation, Curvature Analysis, and Diffusion Limited Aggregation were applied to 2D (growth modeling and contour analysis) and 3D (Micro-CT scan) perspectives of sagittal otoliths of different species.
\end{document}