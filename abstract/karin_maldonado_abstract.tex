\documentclass[a4paper]{article}
%\usepackage{simplemargins}

%\usepackage[square]{natbib}
\usepackage{amsmath}
\usepackage{amsfonts}
\usepackage{amssymb}
\usepackage{graphicx}

\begin{document}
\pagenumbering{gobble}

\Large
\begin{center}
Building a dynamic growth model for natural forests in south-central Chile\\ 

\hspace{10pt}

% Author names and affiliations\
\large
Karin Maldonado, Constanza Weinberger, Natalia Ricote\\

\hspace{10pt}

\small  
Universidad Adolfo Ibáñez\\
karin.maldonado@uai.cl\\

\end{center}

\hspace{10pt}

\normalsize

Global change drivers such as land-use change and climate change (CC) among others, are rapidly altering resource availability, affecting population densities, causing biotic homogenization and the loss of ecological interactions and functional diversity. To make inferences about the breadth of resources or trophic niche width used by an individual, stable isotope analysis (SIA) provides a time-integrated record of dietary inputs (Martinez del Rio et al. 2009). Since animal tissues differ in the rate at which they incorporate the isotopic signatures of dietary items, by sampling different types of tissues in a single individual allows to determine how an animal uses resources over a range of temporal scales. However, the relative contribution of each tissue's isotopic signature to the estimated animal diet is not the same, rather it depends on the dynamic of isotopic incorporation. Our goal is to include the variation of isotopic incorporation rate among tissues for more accurate interpretation of trophic niche width.

On the other hand, numerous studies have demonstrated dramatic shifts in species' phenology - the timing of life-cycle events of plants and animals - as a consequence of CC, because temperature is an essential signal for the development of many species. Those conditions often require that this timing is in synchrony with the phenology of other species, therefore CC may disrupt this synchrony when interacting species exhibit different sensitivity to environmental cues. For example, changes in species' population densities due to CC may have negative consequences on both mutualistic (plants and pollinators or seed dispersers) and antagonistic (plant and herbivores) ecological interactions (Renner and Zohner, 2018). This may affect the expression of species' phenophases, which have cascading community effects detrimental to ecosystem functioning (Visser and Gienapp, 2019). To date, most studies evaluating phenological mismatch have examined the effect of temperature. In addition to increasing temperatures, CC will modify precipitation in many regions of the planet (IPCC, 2013). Although rainfall has been suggested to be a relevant cue for an organism's phenology and strongly impacts community and ecosystem processes preciously few studies have addressed how environmental conditions beyond temperature affects species' phenology (Ovaskainen et al., 2013). We will analyze a collection of long-term phenological data collected from an iconic Long-Term Ecological Research (LTER) station established in Chile in 1989. This LTER is located in the Bosque Fray Jorge National Park, which is a World Biosphere Reserve approximately 150 km from the southern edge of the hyperarid Atacama Desert. The goals of our project are to examine phenological responses to temperature and precipitation on antagonistic (plant and herbivores) ecological interactions, specifically how changes in rodent population densities might affect plants phenophases and viceversa. 
 

\end{document}