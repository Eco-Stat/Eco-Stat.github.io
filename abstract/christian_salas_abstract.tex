\documentclass[a4paper]{article}
%\usepackage{simplemargins}

%\usepackage[square]{natbib}
\usepackage{amsmath}
\usepackage{amsfonts}
\usepackage{amssymb}
\usepackage{graphicx}

\begin{document}
\pagenumbering{gobble}

\Large
 \begin{center}
Building a dynamic growth model for natural forests in south-central Chile\\ 

\hspace{10pt}

% Author names and affiliations
\large
Christian Salas\\

\hspace{10pt}

\small  
Universidad Mayor, Chile\\

\end{center}

\hspace{10pt}

\normalsize
A forest growth model is a system of mathematical equations that describes forest development and mimics some, or all, stages of its dynamics. Growth models are vital for decision-making in forest management because they predict forest development under several silvicultural regimens, which can be linked to forest optimization techniques and management. Forest growth models can also be used for generalizing specific concepts of forest dynamics and dynamics at the ecosystem scale. Although there is an impressive amount of ecological studies on Nothofagus-dominated natural forests in Chile, no growth model is available for assessing the effects of both silvicultural regimes and a changing climate. Furthermore, the scarcity of long-term data and the mixed-species nature of such a complex ecological system offer several challenges in modelling. We aim to build a forest growth model for Nothofagus-dominated forests in south-central Chile by merging statistical and modelling approaches into a mathematical system. We propose a tree-level dynamic system based on differential equations and parameters fitting by non-linear mixed-effects models. We discuss here the system's main model components and our model's assessment.
\end{document}