\documentclass[a4paper]{article}
%\usepackage{simplemargins}

%\usepackage[square]{natbib}
\usepackage{amsmath}
\usepackage{amsfonts}
\usepackage{amssymb}
\usepackage{graphicx}

\begin{document}
\pagenumbering{gobble}

\Large
 \begin{center}
    Comparing two spatial variables with the probability of agreement\\ 

\hspace{10pt}

% Author names and affiliations
\large
Ronny Vallejos$^1$, Jonathan Acosta$^2$, Aaron Ellison$^3$, Felipe Osorio$^4$, Mario de Castro$^5$ \\

\hspace{10pt}

\small  
$^1$ Universidad Técnica Federico Santa María, Chile\\
$^2$ Pontificia Universidad Católica de Chile \\
$^3$ Harvard University, USA\\
$^4$ Universidad Técnica Federico Santa María, Chile\\
$^5$ Universidad de Sao paulo, Brasil\\

\end{center}

\hspace{10pt}

\normalsize

Computing the agreement between two continuous sequences is of great interest in statistics when comparing two instruments or comparing one instrument with a gold standard. The probability of agreement (PA) quantifies the similarity between two variables of interest, and it is useful for accounting what constitutes a practically important difference. In this article we introduce a generalization of the PA for the treatment of spatial variables. Our proposal makes the PA to depend on the spatial lag. Consequently, for isotropic spatial processes, the conditions for which the PA decays as a function of the distance lag are established. The estimation is addressed through a first order approximation that guaranties the asymptotic normality of the sample version of the PA. The sensitivity of the PA is studied for finite sample size, with respect to the covariance parameters. The new method is described and illustrated with real data involving the normalized difference vegetation index (NDVI) associated with forest images, obtained from Harvard Forest, MA, which allows to estimate the probability of agreement between two forest images from the same scene but taken at different times.

\end{document}